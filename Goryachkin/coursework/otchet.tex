\documentclass[a4paper,14pt]{extarticle} %размер бумаги устанавливаем А4, шрифт 14пунктов 
\usepackage[T2A]{fontenc} 
\usepackage[utf8]{inputenc}%включаем свою кодировку: koi8-r или utf8 в UNIX, cp1251 в Windows 
\usepackage[english,russian]{babel}%используем русский и английский языки с переносами 
\usepackage{amssymb,amsfonts,amsmath,mathtext,cite,enumerate,float} %подключаем нужные пакеты расширений 
\usepackage{graphicx} 
\graphicspath{{images/}} 

\usepackage[colorlinks=false,pdfborder={0 0 0}]{hyperref} %использование гиперссылок, colorlinks - цвет текста ссылки, pdfborder - окантовка. 

\usepackage{alltt} 
\usepackage{fancyvrb} 
%шрифт Times New Roman 
%\usepackage{fontspec} 
%\setmainfont{Times New Roman} 
%\setallmainfonts{Times New Roman} 

\usepackage{titlesec} 

\makeatletter 
\renewcommand{\@biblabel}[1]{#1.} % Заменяем библиографию с квадратных скобок на точку: 
\makeatother 

\usepackage{graphicx} 
\graphicspath{{pictures/}}

\usepackage{geometry} % Меняем поля страницы 
\geometry{left=3cm}% левое поле 
\geometry{right=15mm}% правое поле 
\geometry{top=2cm}% верхнее поле 
\geometry{bottom=2cm}% нижнее поле 
\linespread{1.5} 

\usepackage{indentfirst} % отделять первую строку раздела абзацным отступом 
\setlength\parindent{5ex} 
\usepackage{tikz} 
\usepackage{pgfplots} 
%links 
\usepackage{url} 

\usepackage[tableposition=top,singlelinecheck=false, justification=centering]{caption} 
\usepackage{subcaption} 

% маркированные списки 
\renewcommand{\labelitemi}{--} 
\renewcommand{\labelitemii}{--} 
% нумерованные списки 
\renewcommand{\labelenumi}{\asbuk{enumi})} 
\renewcommand{\labelenumii}{\arabic{enumii})} 

% номер сноски со скобкой 
\renewcommand*{\thefootnote}{\arabic{footnote})} 
\renewcommand{\footnoterule}{% 
\kern -3pt 
\hrule width 40mm height .4pt 
\kern 2.6pt 
} 

%иллюстрации и таблицы 
\DeclareCaptionLabelFormat{gostfigure}{Рисунок #2} 
\DeclareCaptionLabelFormat{gosttable}{Таблица #2} 
\DeclareCaptionLabelSeparator{gost}{~---~} 
\captionsetup{labelsep=gost} 
\captionsetup*[figure]{labelformat=gostfigure} 
\captionsetup*[table]{labelformat=gosttable} 
\renewcommand{\thesubfigure}{\asbuk{subfigure}} 

\usepackage{tocloft} 
\renewcommand{\cftsecleader}{\cftdotfill{\cftdotsep}} 
%\renewcommand{\cfttoctitlefont}{\Large\filcenter} 
%\setcounter{page}{3} %нумерация страниц с 3 
\addto\captionsrussian{\renewcommand\contentsname{СОДЕРЖАНИЕ}} 
\addto\captionsrussian{\renewcommand\refname{СПИСОК ИСПОЛЬЗОВАНЫХ ИСТОЧНИКОВ}} 

\usepackage{listings} 
\lstset{ 
frame=single, 
breaklines=true 
} 
\author{Д.С. Горячкин} 
\title{Список дел(to-do list)} 
\begin{document} 
\begin{titlepage} 
\begin{center} 

ГУАП\\ 
КАФЕДРА № 52\\ 
\vspace{2cm} 

\begin{flushleft} 
ПРЕПОДАВАТЕЛЬ 
\begin{tabular}{|l|l|l|} 
\hline 
доц., канд. техн. наук & & Е.М. Линский\\ 
\hline 
должность, уч. степень, звание & подпись, дата & инициалы, фамилия\\ 
\hline 
\end{tabular} 
\end{flushleft} 

\vspace{3cm} 

{\Large ОТЧЁТ О КУРСОВОЙ РАБОТЕ\\} 
\vspace{0.3cm} 
{\Large Список дел(to-do list)} 

\vspace{0.7cm} 

\begin{flushleft} 
по курсу: ТЕХНОЛОГИИ ПРОГРАММИРОВАНИЯ 
\end{flushleft} 

\vspace{5cm} 

\begin{flushleft} 
РАБОТУ ВЫПОЛНИЛ 
\begin{tabular}{|l|l|l|} 
\hline 
СТУДЕНТ ГР. № 5723 & & Д.С. Горячкин\\ 
\hline 
& подпись, дата & инициалы, фамилия\\ 
\hline 
\end{tabular} 
\end{flushleft} 

\vspace{2cm} 

Санкт-Петербург 2020

\end{center} 
\end{titlepage} 
\renewcommand{\chaptername}{Раздел} 
\renewcommand{\figurename}{Рисунок} 

\setcounter{page}{2} 
\large\bf {{Содержание}}
\begin{itemize}
    \item 1. Спецификация к курсовой работе.............................3
    \item 2. Архитектура проекта.................................................7
\end{itemize}

\newpage
\section{Спецификация к курсовой работе} 
\subsection{Описание}
\begin{itemize}
    \item{Cписок дел - это ежедневник в который записываются задачи, которые вам нужно выполнить}
    \item{Данный список дел будет написан на Java Servlet}
\end{itemize}
\subsection{Возможности} 
\begin{itemize} 
    \item{Поддерживается регистрация и вход для каждого пользователя}
    \item{Главная страница:} 
    \begin{itemize} 
        \item{Создание списка дел}
        \item{Редактирование списка}
        \item{Sign out}
    \end{itemize} 
    \item{Профиль:}
    \begin{itemize}
        \item{У всех пользователей показывается их список дел}
    \end{itemize}
\subsection{Инструкция пользователя}
    \begin{figure}[H]
        \centering\includegraphics[width=\linewidth]{qwer.png}
        \caption{Выбор между регистрацией и входом}
    \end{figure}
    \item{Вход и регистрация}
    \begin{figure}[H]
        \centering\includegraphics[width=\linewidth]{qwer1.png}
        \caption{Регистрация}
    \end{figure}

    \begin{itemize}
        \item{Незарегестрированный пользователь не сможет войти в систему, ему будет предложено попробывать снова или зарегистрироваться}
    \end{itemize}
    \item{Добавление дела в список:} 
    \begin{figure}[H]
        \centering\includegraphics[width=\linewidth]{qwer2.png}
        \caption{Добавление дела}
    \end{figure}
    \begin{itemize} 
        \item{Пользователю будет предложена форма, в которой можно указать дело которое нужно выполнить, оно будет добавлено вниз под формой} 
    \end{itemize}
    \item{Удаления дела из списка:}
    \begin{figure}[H]
        \centering\includegraphics[width=\linewidth]{qwer3.png}
        \caption{Удаление дела}
    \end{figure}
    \begin{itemize} 
        \item{Пользователю может удалить дело нажав на крести рядом с ним, либо отметив как выполненое и нажать на кнопку удаления всех выполненых дел} 
    \end{itemize}
 \item{Права пользователя:}
    \begin{itemize}
        \item{У пользователя есть возможность добавления дела в список, просмотр списка дел,  есть возможность удаления дел из списка, выход из аккаунта}
    \end{itemize}
\end{itemize}
\newpage
\section{Архитектура проекта}
Архитектура проекта состоит из одного пакета. Рассмотрим состовляющие пакета:
    \item {Пакет servlets - содержит сервлеты.}
    \begin{itemize}
        \item\bf{public class DeleteServlet extends HttpServlet} - сервлет для редактирования списка 
        \item\bf{public class SignoutServlet extends HttpServlet} - сервлет для выхода пользователя из аккаунта
        \item\bf{public class SignupServlet extends HttpServlet} - сервлет для регистрации пользователя 
        \item\bf{public class SinginServlet extends HttpServlet} - сервлет для входа в аккаунт
        \item\bf{public class ToDoServlet extends HttpServlet} - сервлет для создания списка
    \end{itemize}
\end{document}