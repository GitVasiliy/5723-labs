\documentclass[a4paper,14pt]{report}

\usepackage[utf8]{inputenc}
\usepackage[russian]{babel}

\begin{document}

	\begin{titlepage}
			\begin{center}
				СПЕЦИФИКАЦИЯ К КУРСОВОЙ РАБОТЕ
				\par
				\vspace{4cm}
				\huge
				{СЕТЕВАЯ ИГРА В КРЕСТИКИ-НОЛИКИ\parНА ПОЛЕ 10х10\par}
				\vspace{2cm}
				\normalsize по курсу: Технологии программирования
			\end{center}
		\vspace{5cm}
		\begin{flushleft} 
			Выполнила работу:\par
			студент группы 5723\par
			Глушенкова А.Ю.
		\end{flushleft}
		\begin{center}
			\vspace{3cm}
			Санкт-Петербург 2019
		\end{center}
	\end{titlepage}

	\newpage
	\section*{Фукнциональная спецификация}
	\paragraph{\largeБазовая версия\\}
	\normalsize
		Сетевая игра крестики-нолики будет две реализации: серверную и клиентскую, которые обе используют TCP соединение. Клиент будет подключаться к серверу с определенным ip. Игра имеет графический интерфейс.\\ 
		Клиенту при заходе на сервер будет предложена аутентификация или регистрация, если у него нет аккаунта. Каждый игрок должен иметь уникальный логин.\\
		Игрок может либо создать новую доску, либо присоединиться к уже созданной. Игра ведется, пока у одного из игрока не будет 5 одинаковых символов в ряд или по диагонали. Для каждого игрока ведется подсчет очков: 3 - победа, 2 - ничья, 1 - проигрыш. После окончания игры, игроку будет предложено начать еще одну или выйти. 
		
		
	\vspace{1cm}
	\paragraph{\large\\}
	\normalsize

	

\end{document}