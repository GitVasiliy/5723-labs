\documentclass[a4paper,14pt]{extarticle} %размер бумаги устанавливаем А4, шрифт 14пунктов
\usepackage[T2A]{fontenc}
\usepackage[utf8]{inputenc}%включаем свою кодировку: koi8-r или utf8 в UNIX, cp1251 в Windows
\usepackage[english,russian]{babel}%используем русский и английский языки с переносами
\usepackage{amssymb,amsfonts,amsmath,mathtext,cite,enumerate,float} %подключаем нужные пакеты расширений
\usepackage{graphicx} %хотим вставлять в диплом рисунки?
\graphicspath{{images/}}%путь к рисункам

\usepackage[colorlinks=false,pdfborder={0 0 0}]{hyperref} %использование гиперссылок, colorlinks - цвет текста ссылки, pdfborder - окантовка.

\usepackage{alltt}
\usepackage{fancyvrb}
%шрифт Times New Roman
%\usepackage{fontspec}
%\setmainfont{Times New Roman}
%\setallmainfonts{Times New Roman}

\usepackage{titlesec}

\makeatletter
\renewcommand{\@biblabel}[1]{#1.} % Заменяем библиографию с квадратных скобок на точку:
\makeatother

\usepackage{graphicx}
\graphicspath{{pictures/}}

\usepackage{geometry} % Меняем поля страницы
\geometry{left=3cm}% левое поле
\geometry{right=15mm}% правое поле
\geometry{top=2cm}% верхнее поле
\geometry{bottom=2cm}% нижнее поле
\linespread{1.5}

\usepackage{indentfirst} % отделять первую строку раздела абзацным отступом
\setlength\parindent{5ex}
\usepackage{tikz}
\usepackage{pgfplots}
%links
\usepackage{url}

\usepackage[tableposition=top,singlelinecheck=false, justification=centering]{caption}
\usepackage{subcaption}

% маркированные списки
\renewcommand{\labelitemi}{--}
\renewcommand{\labelitemii}{--}
% нумерованные списки
\renewcommand{\labelenumi}{\asbuk{enumi})}
\renewcommand{\labelenumii}{\arabic{enumii})}

% номер сноски со скобкой
\renewcommand*{\thefootnote}{\arabic{footnote})}
\renewcommand{\footnoterule}{%
\kern -3pt
\hrule width 40mm height .4pt
\kern 2.6pt
}

%иллюстрации и таблицы
\DeclareCaptionLabelFormat{gostfigure}{Рисунок #2}
\DeclareCaptionLabelFormat{gosttable}{Таблица #2}
\DeclareCaptionLabelSeparator{gost}{~---~}
\captionsetup{labelsep=gost}
\captionsetup*[figure]{labelformat=gostfigure}
\captionsetup*[table]{labelformat=gosttable}
\renewcommand{\thesubfigure}{\asbuk{subfigure}}

\usepackage{tocloft}
\renewcommand{\cftsecleader}{\cftdotfill{\cftdotsep}}
%\renewcommand{\cfttoctitlefont}{\Large\filcenter}
%\setcounter{page}{3} %нумерация страниц с 3
\addto\captionsrussian{\renewcommand\contentsname{СОДЕРЖАНИЕ}}
\addto\captionsrussian{\renewcommand\refname{СПИСОК ИСПОЛЬЗОВАНЫХ ИСТОЧНИКОВ}}

\usepackage{listings}
\lstset{
frame=single,
breaklines=true
}
\author{С. Е. Богачев}
\title{Многопользовательский Paint}
\begin{document}
\begin{titlepage}
\begin{center}

ГУАП\\
КАФЕДРА № 52\\
\vspace{2cm}

\begin{flushleft}
ПРЕПОДАВАТЕЛЬ
\begin{tabular}{|l|l|l|}
\hline
доц., канд. техн. наук & & Е.М. Линский\\
\hline
должность, уч. степень, звание & подпись, дата & инициалы, фамилия\\
\hline
\end{tabular}
\end{flushleft}

\vspace{3cm}

{\Large СПЕЦИФИКАЦИЯ\\}
\vspace{0.3cm}
{\Large Web-каталог для магазина.}

\vspace{0.7cm}

\begin{flushleft}
по курсу: ТЕХНОЛОГИИ ПРОГРАММИРОВАНИЯ
\end{flushleft}

\vspace{5cm}

\begin{flushleft}
РАБОТУ ВЫПОЛНИЛ
\begin{tabular}{|l|l|l|}
\hline
СТУДЕНТ ГР. № 5723 & & С.Е. Богачев\\
\hline
& подпись, дата & инициалы, фамилия\\
\hline
\end{tabular}
\end{flushleft}

\vspace{2cm}

Санкт-Петербург 2019

\end{center}
\end{titlepage}
\renewcommand{\chaptername}{Раздел}
\renewcommand{\figurename}{Рисунок}

\begin{center}
\huge \bf Web-каталог для магазина
\end{center}

\setcounter{page}{2}
\section*{Возможности}
WEB-каталог это программа, которая используется для удобного отображения товаров в виде каталога в интернете.Используеться динамичекий рендеринг. Программа загружает из файла каталог товаров и их описание, а затем отображает их в виде интернет-страницы.В программе доступны следующие функции:
\item{Возможности администратора:}
\begin{itemize}

\item{Поле для входа администратора.}
\item{Добавление товар.}

\item{Редактирование товар.}
\begin{itemize}
\item{Удаление товар.}
\item{Загрузка картинки товара с компьютера.}
\item{Загружать количество товара, имя,
цену.}
\end{itemize}
\end{itemize}
\item{Возможности пользователя:}
\begin{itemize}
\item{Просмотрт фотографий каждого товара.}
\item{Сортировать товар по алфовиту, цене и дате добавления.}
\item{Добавление товар в корзину.}
\item{Удаление товара из корзины.}
\item{Просмот товар в корзине.}
\item{Просмотр товара по катигории.}
\item{Просмотр количества товара, имя, цену.}
\end{itemize}


\section*{Макет приложения}

При запуске пользователь видит главное окно страницы, где представлены все товары. Вверху страницы представлены ссылки для входа администратора, перехода в каталог и перехода в корзину, ниже будут расположены сортировки по имени, цене и размеру. Слева разные категории товаров

\begin{figure}[h]
\centering
\includegraphics[width=0.8\linewidth]{1.jpg}
\caption{\bf Главная страница.}
\end{figure}

Пользователю можно выбрать: добавить товар или перейти к корзине.
\newpage
Пользователь может просмотреть какие товары находяться у него в корзине.
\begin{figure}[h]
\centering
\includegraphics[width=0.8\linewidth]{3.jpg}
\caption{\bf Корзина.}
\end{figure}

\newpage
Администратор может добавлять/удалять товар или изменять количество товаров.
\begin{figure}[h]
\centering
\includegraphics[width=0.8\linewidth]{2.jpg}
\caption{\bf Вход для администратора.}
\end{figure}

\newpage
\end{document}