\documentclass[12pt,a4paper]{article}
\usepackage{cmap}
\usepackage[utf8]{inputenc}
\usepackage[russian]{babel}
\begin {document}
\thispagestyle{empty}
\begin{center}
ГУАП

КАФЕДРА №52
\end{center}

\vspace{30mm}
ПРЕПОДАВАТЕЛЬ

\begin{tabular}{ |c|c|c| }
\hline
доц., к. т. н. & & Е.М. Линский \\\hline
должность, уч. ст., звание & дата, подпись & инициалы, фамилия \\\hline
\end{tabular}

\vspace{30mm}
\begin{center}
СПЕЦИФИКАЦИЯ

ПЕРЕСЫЛКА ФАЙЛОВ НА ДРУГОЙ КОМПЬЮТЕР
\end{center}

\vspace{10mm}
по курсу: ТЕХНОЛОГИИ ПРОГРАММИРОВАНИЯ

\vspace{50mm}
\hspace{8mm}РАБОТУ ВЫПОЛНИЛ
\begin{center}
\begin{tabular}{ |c|c|c| }
\hline
Студент гр. 5723 & & Л.А.Кормилицына \\\hline
& дата, подпись & инициалы, фамилия \\
\hline
\end{tabular}
\vspace{15mm}

Санкт-Петербург 2019
\end{center}

\newpage
\begin{center}
\Large{\bf Пересылка файла на другой компьютер }
\end{center}
\setcounter{page}{2}

\section*{Возможности программы}
\begin{itemize}
\item Передача текстовых сообщений между сервером и клиентом;
\item Клиент получает имя файла при старте и посылает файл клиенту:
Передача файлов по TCP между сервером и клиентом, где сервер сохраняет полученный файл в заданный каталог с указанием нового имени файла и расширения;
\item Cмена и установка имен клиента и сервера при обращении друг к
другу;

\end{itemize}
\newpage

\section*{Описание работы программы}

Данная программа будет реализовывать создание сервера и  устанавливать соединение сервер-клиент, получая на запуске   свободный порт и адресс машины в виде аргументов, на которой расположен сервер . После соединения (если оно прошло успешно) программа выводит меню с коммандами на экран:
\begin{itemize}
\item Commands list: @help
\item Set users name: @name Luba
\item Send text message: Hello, world !
\item Send file: @send filename
\item Cancel command: @cancel
\item Quit program: @quit
\end{itemize}
После ввода комманды " @send имя-файла расширение-файла " на экране принимающей файл стороны появляется сообщение:
\begin{itemize}
\item Enter the directory to save:
\end{itemize}
Вводим директорию для сохранения файла с указанием имени файла и
расширения. Далее на сервере и клиенте отображается процесс пересылки
файла и сообщение о успешной пересылке файла. Во время передачи файлов будет отображаться строка загрузки/передачи файла
(ProgressBar)

 [\#\#\#.......]
 
 Где символ "\#" будет процент загруженного файла, а "." будет показывать процент еще не загруженного файла. ProgressBar будет отображать текущий прогресс отпраки/загрузки и будет динамически обновляться в соответствии с количеством переданных байт/к общему размеру файла.

\begin{itemize}
\item Begin receiving file
\item End receiving file.
\item Complete.
\end{itemize}

\newpage
\begin{center}
\Large{\bf Пример работы программы }
\end{center}

\Large{\bf Со стороны сервера }
\\System:
\\Commands list: @help
\\Set users name: @name Luba
\\Send text message: Hello, world !
\\Send file: @send filename
\\Cancel command: @cancel
\\Quit program: @quit
\\SYSTEM: /127.0.0.1 connected.
\\@send message.txt
\\
\\[\#\#\#\#\#.......]

\Large{\bf Со стороны клиента }
\\System:
\\Commands list: @help
\\Set users name: @name Luba
\\Send text message: Hello, world !
\\Send file: @send filename
\\Cancel command: @cancel
\\Quit program: @quit
\\SYSTEM: Enter the directory to save:
\\D:\textmessage.txt
\\SYSTEM: Begin receiving file.
\\
\\[\#\#\#\#\#.......]
\\SYSTEM: End receiving file.
\\SYSTEM: Complete.

\end{document}