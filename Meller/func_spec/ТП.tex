\documentclass[12pt,a4paper]{article}
\usepackage{cmap}
\usepackage[utf8]{inputenc}
\usepackage[russian]{babel}
\begin {document}
\thispagestyle{empty}
\begin{center}
ГУАП

КАФЕДРА №52
\end{center}

\vspace{30mm} 
ПРЕПОДАВАТЕЛЬ

\begin{tabular}{ |c|c|c| }
\hline
доц., к. т. н. &  & Е.М. Линский \\\hline
должность, уч. ст., звание & дата, подпись & инициалы, фамилия \\\hline
\end{tabular}

\vspace{30mm}
\begin{center}
СПЕЦИФИКАЦИЯ

СИСТЕМА ОБМЕНА ФАЙЛАМИ ЧЕРЕЗ ВЕБ
\end{center}

\vspace{10mm}
по курсу: ТЕХНОЛОГИИ ПРОГРАММИРОВАНИЯ

\vspace{50mm}
\hspace{8mm}РАБОТУ ВЫПОЛНИЛ
\begin{center}
\begin{tabular}{ |c|c|c| }
\hline
Студент гр. 5723 &  & А.А.Меллер \\\hline
 & дата, подпись & инициалы, фамилия \\
\hline
\end{tabular}
\vspace{15mm}

Санкт-Петербург 2019
\end{center}

\newpage
\begin{center}
   \Large{\bf Система обмена файлами через веб } 
\end{center}

Данная программа предоставляет возможность пользователям обмениваться файлами посредством сервера. Каждый пользователь имеет доступ не только к своей директории, но и к директориям, которые открыты ему другими пользователями. Админ имеет доступ ко всем директориям. 

Интерфейс пользователя будет представлять из себе несколько веб-страниц. Основные страницы: страница регистрации, страница со списком видимых директорий и страница со списком доступных файлов в директории.

Для обеспечения хотя бы минимального уровня безопасности будет организована аутентификация по логину и паролю.

\begin{center}
   \Large{\bf Возможности } 
\end{center}

\begin{enumerate}
\item Регистрация нового пользователя;
\item Скачивать файлы из открытых пользователю директорий;
\item Загружать файлы в свою директорию;
\item Делиться своей директорией с другими пользователями;
\item Искать файлы по открытым ему директориям;
\item При открытии директорий другим пользователям можно указать модификатор доступа: "Только чтение" и "Чтение и редактирование";
\item Директории имеющие модификатор доступа "Только чтение" невозможно редактировать;
\item Можно сгенерировать прямую ссылку на скачивание файла из открытых пользователю директорий;
\item Если пользователь является админом, то он имеет полный доступ (с возможностью изменения всех элементов) ко всем директориям;
\item Админ так же может производить поиск по всем директориям.
\end{enumerate}

\newpage
\begin{center}
   \Large{\bf Инструкция пользователя } 
\end{center}

После открытия сайта необходимо зарегистрировать новый аккаунт или войти в уже существующий. После входа появляется список доступных директорий (каталогов). Зайдя в любую из них, появляются доступные для скачки файлы, а так же кнопка "Загрузить" для загрузки нового файла (доступно только для своей собственной директории). Вернуться назад можно будет нажатием кнопки "Назад". Для поиска файла необходимо заполнить форму вверху страницы и нажать кнопку "Найти".

Если пользователь хочет поделиться с кем-то своим каталогом он может нажать кнопку "Поделиться" и выбрать имя другого пользователя в списке, а также модификатор доступа. Если необходимо просто дать другому пользователю доступ к файлам (без возможности изменять и удалять файлы), то следует указать модификатор доступа "Только чтение". Если мы открываем полный доступ со всеми возможностями, то следует выбрать модификатор "Чтение и редактирование".  Отметим, что в каталогах, которыми с пользователем кто-то поделился, нельзя производить изменения. То есть они доступны только для скачивания и просмотра списка файлов.

У админа после входа в аккаунт появляются сразу все существующие директории.Соответственно админу доступны все действия обычного пользователя, но с доступом ко всем директориям сразу. Стоит отметить, что для админа все директории имеют по умолчанию модификатор доступа "Чтение и редактирование". 

\end{document}